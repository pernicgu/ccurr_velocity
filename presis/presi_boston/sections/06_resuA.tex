% 06 results A
\begin{frame}{Could we simply use approximations?}
  
  Simple ways out to not use all these complicated measures:
  \begin{itemize}\setlength{\itemsep}{0.5em}
  \item \textbf{Coin days destroyed}
  \item Approximate coin-turnovers\footnote{\emph{\scriptsize[Smith, Reginald D: Bitcoin Average Dormancy: {A} Measure of Turnover and Trading Activity. 2017.]}}% \cite{smith2017bitcoin}
  \item Ratio: Raw on-chain transaction volume to total coin supply
  \end{itemize}

\end{frame}

% \begin{frame}{Coin Days Destroyed VS. \(\VCircWbP\) with $p = 1$ day}
%   \vspace{0.75cm}
%   \begin{figure}[h]
%     \centering
%     \inputTikz{desc_all_app_norm}
%     \caption{Data for Bitcoin, 06/2013 to 06/2019}
%   \end{figure}
%   \vspace{2em}

%   How can we compare these lines?
% \end{frame}

% \begin{frame}{Velocity measures over time with $p = 1$ day}
% 	\vspace{0.75cm}
% 	\begin{figure}[h]
% 		\centering
% 		\inputTikz{desc_all_est_norm}
% 	%    \caption{Velocity measures (normalized) over time.}
% 	\end{figure}
% \end{frame}

%\begin{frame}{Coin Days Destroyed VS. $\VCirc$}
%\vspace{0.75cm}
%	\begin{figure}[h]
%		\centering
%		\inputTikz{desc_all_cmp_norm}
%		%    \caption{.}
%	\end{figure}
%\end{frame}

\begin{frame}{Which one should I use?}	
	\begin{alertblockc}{}{jwigreige!100!white}%
		Coin days destroyed is almost in all constellations yielding the largest deviations
	\end{alertblockc}
	\begin{alertblockc}{}{jwilightgreen!90!white}%
		The trivial measure is almost always significantly lowest
	\end{alertblockc}
	
	\textbf{\emphtext{How did we test?}}
	\begin{itemize}  
		\item Bitcoin, daily, 06/2013-06/2019
		\item Normalization / Standardization
		\item Mean Absolute Errors / Mean squared Errors
		\item Raw / First differences
		\item Model Confidence Set test for significance\footnote{\emph{\scriptsize[Hansen et al.: {T}he model confidence set. 2011.]}}
	\end{itemize}
\end{frame}

%\begin{frame}{Model Confidence Test---Significance of results}
%
%We compare models \(i\) and \(j\) for all $ i,j \in M $ where $i \neq j$. \newline  %
%\smallskip
%Loss functions are MAE and MSE. \newline%
%\smallskip
%Thus (exemplified for MSE)
%  \begin{align*}
%    d_{ij \perd} = \bigl( V^{\est}_\perd
%    - V^{\app}_{i \perd} \bigr)^{2}
%    - \bigl( V^{\est}_\perd
%    - V^{\app}_{j \perd} \bigr)^{2}.%
%  \end{align*}
%  % \begin{align*}
%  %   d_{ij \perd} = \bigl| V^{\est}_\perd
%  %   - V^{\app}_{i \perd} \bigr|
%  %   - \bigl| V^{\est}_\perd
%  %   - V^{\app}_{j \perd} \bigr|.%
%  % \end{align*}
%  The \emphtext{relative performance} of model $i$ compared to all other models then is
%  \begin{align*}
%    d_{i \boldsymbol{\cdot}} = \frac{1}{m-1} \sum_{j \in M \setminus i} d_{ij},%.
%  \end{align*}
%  with $i = 1,...,m$. %
%  The null hypothesis states %
%  \begin{align*}
%    H_{0M}: E(d_{i \boldsymbol{\cdot}}) = 0,\ \forall i \in M. %
%  \end{align*}  
%\end{frame}

% \begin{frame}{What are the approximators approximating? And how well?}
  
%   \begin{center}
%     \begin{table}
%       \caption{MAE and MSE between velocity approximation methods compared to estimation methods. Approximators in superiority set of MCS test (1\% sig. levels) are marked with \(\dag\).}
%       \inputTable{0.55}{appVSest_errors_w_mcs}
%     \end{table}
%     \vspace{-4mm}
%   \end{center}

% \end{frame}